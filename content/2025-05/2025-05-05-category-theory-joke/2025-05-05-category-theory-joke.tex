\documentclass{article}
\usepackage{amsthm, amsmath, amssymb}
\usepackage{hyperref}
\usepackage[margin=1.0in]{geometry}

\newtheorem{theorem}{Theorem}
\newtheorem{joke}{Joke}
\newtheorem{definition}{Definition}
\newtheorem{observation}{Observation}

\title{The Comprehensive Guide to Category Theory Jokes}
\author{Claude Comedian}
\date{May 5, 2025}

\begin{document}

\maketitle

\begin{abstract}
    This note presents a meta-analysis of humor in the context of category theory, examining the curious phenomenon that category theory jokes are rarely funny. We develop a theoretical framework for understanding this humor gap and provide examples that demonstrate why abstraction and humor often form a highly non-commutative pair.
\end{abstract}

\section{Introduction}

Category theory stands as one of the most abstract domains of mathematics, providing a language to describe mathematical structures and relationships between them. Its practitioners delight in concepts like functors, natural transformations, and the ever-popular Yoneda lemma. However, this same community struggles mightily with a simple task: being funny on purpose.

\begin{observation}
    The humor value of a mathematics joke is inversely proportional to the level of abstraction in the underlying mathematical concept.
\end{observation}

\section{The Joke Category $\mathbf{Joke}$}

We begin by formalizing the structure of mathematical jokes.

\begin{definition}
    The category $\mathbf{Joke}$ consists of:
    \begin{itemize}
        \item Objects: Mathematical concepts that form the subject of jokes
        \item Morphisms: Humorous connections between concepts
        \item Composition: Building more complex jokes from simpler ones
        \item Identity: The ever-reliable "Why did the mathematician cross the road?" template
    \end{itemize}
\end{definition}

\begin{theorem}[Humor Preservation Failure]
    There is no faithful functor $F: \mathbf{Cat} \to \mathbf{Joke}$ from the category of small categories to the category of jokes.
\end{theorem}

\begin{proof}
    Assume by contradiction that such a faithful functor $F$ exists. Then for any concept in category theory, there would exist a genuinely funny joke about it. However, extensive empirical evidence contradicts this hypothesis, as seen in the following examples.
\end{proof}

\section{Attempted Category Theory Jokes}

\begin{joke}
    Q: Why did the functor turn back?\\
    A: It forgot its adjoint!
\end{joke}

\begin{observation}
    The joke attempts to anthropomorphize a functor while making a pun on "adjoint" vs. "a joint." The humor fails because functors gain nothing from anthropomorphization, and the pun is too strained.
\end{observation}

\begin{joke}
    An initial object walks into a bar. Everyone ignores it because they know it'll come to them eventually.
\end{joke}

\begin{observation}
    This joke requires understanding that an initial object has a unique morphism to every object. The humor is undercut by the fact that this property, while mathematically elegant, lacks inherent comedic potential.
\end{observation}

\section{The Yoneda Perspective on Humor}

The Yoneda lemma tells us that an object is completely determined by the morphisms into it. Similarly, we propose:

\begin{theorem}[Yoneda Humor Lemma]
    A category theory joke is completely determined by how unfunny people find it across all possible audiences.
\end{theorem}

\begin{joke}
    I told a joke about the Yoneda lemma, but nobody laughed. I guess they didn't have a natural transformation of humor.
\end{joke}

\begin{observation}
    The above joke demonstrates our theorem perfectly.
\end{observation}

\section{The Terminal Object of Patience}

\begin{definition}
    The Terminal Object of Patience is the point at which a mathematician gives up trying to explain why a category theory joke should be funny.
\end{definition}

\begin{joke}
    Two functors walk into a bar. The bartender asks, "Why the long face?"\\
    The first functor says, "I'm not naturally isomorphic to myself."\\
    The second functor says, "He's lying. He's just not faithful."\\
    [This joke continues for 17 more lines, explaining increasingly obscure category theory concepts without reaching a punchline]
\end{joke}

\begin{observation}
    By the time one finishes explaining this joke, the Terminal Object of Patience has been reached by both the teller and the audience.
\end{observation}

\section{Comparative Humor Analysis}

To better understand why category theory jokes fail, we compare them with jokes from other mathematical fields:

\begin{joke}[Number Theory]
    Why did $7$ eat $9$?\\
    Because you're supposed to eat $3$ squared meals a day!
\end{joke}

\begin{joke}[Topology]
    A topologist is someone who doesn't know the difference between a coffee mug and a donut.
\end{joke}

\begin{joke}[Category Theory]
    A natural transformation walks into a bar. Actually, it doesn't "walk in" so much as there exists a morphism in the category $\mathbf{Bar}$ such that the appropriate diagram commutes up to isomorphism in a way that preserves the structure of...
    [joke continues for three more pages]
\end{joke}

\begin{observation}
    The category theory joke fails because it prioritizes precise mathematical description over the timing and surprise essential to humor.
\end{observation}

\section{Conclusion}

Our research confirms that category theory and humor form a highly non-commutative pair—applying category theory before humor results in mathematical precision but comedic failure, while applying humor before category theory prevents the explanation from ever reaching category theory at all.

Perhaps the final joke is on us: in trying to formalize why category theory jokes aren't funny, we've produced yet another unfunny exploration of category theory.

\begin{observation}
    The set of all funny category theory jokes and the empty set are naturally isomorphic.
\end{observation}

\bibliographystyle{alpha}
\bibliography{category_jokes}

\end{document}